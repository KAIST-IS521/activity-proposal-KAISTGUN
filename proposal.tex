\documentclass[a4paper, 11pt]{article}

\usepackage{kotex} % Comment this out if you are not using Hangul
\usepackage{fullpage}
\usepackage{hyperref}
\usepackage{amsthm}
\usepackage[numbers,sort&compress]{natbib}

\theoremstyle{definition}
\newtheorem{exercise}{Exercise}

\begin{document}
%%% Header starts
\noindent{\large\textbf{IS-521 Activity Proposal}\hfill
                \textbf{김건우}} \\
         {\phantom{} \hfill \textbf{KAISTGUN}} \\
         {\phantom{} \hfill Due Date: April 15, 2017} \\
%%% Header ends


\section{Goals}
이 Activity의 메인 goal은 실제 exploit이 어떤 방식으로 동작하는지 학습하는 것이다.

\section{Activity Overview}

각 학생들은 자신들만의 서비스를 제공하 Exploitable한 서버를 개발한다.  
vulnerability는 널리 알려진 취약점을 대상으로 한다. (ex. Buffer overflow, Heap overflow, Heap spray...)
제작자는 서버와 함께 exploit할 수 있는 코드와 write-up을 제출한다. 이 액티비티는 그동안 학생들이 실습했던 것들을 종합하고 (서버 통신 및 프로그램 개발) 학기 마지막의 mini-CTF를 연습할 수 있는 좋은 계기가 될 것이다.

(Inspired by DEFCON CTF)

\section{Exercises}

\begin{exercise}

  학생들이 자신만의 간략한 서비스를 제공하는 서버를 만든다. 복잡한 서버가 아닌 이전 실습에 사용되었던 Vulnet과 같은 로그인+ 쉘 제공 과 같은 nc로 접속이 가능한 간략한 기능만 가지고있다. 다만 이 서버는 vulnet과 같이 상당히 취약하며, 널리 알려진 exploit (Exploit 목록은 미리 주어진다).을 통해 특정 flag를 얻을 수 있다. 또한 이 서버에는 백도어가 존재한다.

\end{exercise}

\begin{exercise}

 서버소스 코드를 제출한 뒤, 가장 교육적 효과가 높은 프로그램 하나를 선정해 가점을 주고 그 문제를 대상으로 CTF를 한다. 문제를 푼 사람은 write-up을 제출하며, 정해진 시간 내 가장 먼저 푼 사람에게 가점을 준다. 만약 백도어까지 발견할 경우 역시 가점을 준다.

\end{exercise}

\section{Expected Solutions}


1. 서버 코드\\ 모든 학생들은 자신만의 Unique한 서비스를 제공하는 솔루션을 제출한다.\\
2. CTF\\학생들은 Write-up을 제공하되, 출제자가 원하는 방법으로 플래그를 얻었을 경우를 만점으로 환산한다. 또한 가장 먼저 성공한 사람과 백도어를 발견한 사람에게 가점을 준다.


\bibliography{references}
\bibliographystyle{plainnat}

\end{document}